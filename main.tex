\documentclass[12pt,a4paper]{report}
\usepackage{graphicx} % For including images
\usepackage{tocloft} % Optional: Customize the table of contents
\usepackage[a4paper, margin=2cm]{geometry} % Adjust margin 
\setcounter{secnumdepth}{3} % Allows subsubsections to be numbered
\setcounter{tocdepth}{3} % Includes up to subsubsections in the TOC

\begin{document}

% Title Page
\begin{titlepage}
    \centering
    \begin{flushright}
        \centering
        \includegraphics[width=2cm]{images/thd.png} % Adjust width as needed
        \hspace{0cm} % Space between images
        \includegraphics[width=2cm]{images/ipb.jpg} % Adjust width as needed
    \end{flushright}
    {\huge\bfseries Development of an R Toolbox for Near-Infrared Spectroscopy Data Processing \\
    and Analysis of Plant Metabolic Phenotypes\par}
    \vspace{2cm}
    {\LARGE \textsc{Deggendorf Institute of Technology}\par}
    \vspace{1cm}
    {\Large \textsc{MSc. Life Science Informatics}\par}
    \vspace{1.5cm}
    {\Large\itshape Methun George\par}
    \vfill
    Supervised by\par
    PD Dr. habil. rer. nat. Steffen Neumann\par
    Prof. Dr. Melanie Kappelmann-Fenzl
    \vfill
    {\large \today\par}
\end{titlepage}

% Table of Contents
\newpage
\tableofcontents
\newpage

% Chapters and Sections

\chapter{Introduction}
The advancement and widespread use of heigh-throughput experimental 
technologies in the field of plant biology have introduced significant 
challenges in managing and analysing the vast datasets effectively. Addressing
these challenges require innovative methods that maximize the data utility 
while mimnimizing computational inefficiencies and resource consumption, ensuring
robust insights into complex biological systems (https://www.frontiersin.org/research-topics/6856/machine-learning-in-plant-science/articles). 
\section{Related Work}
The related work include, the fillowings

\chapter{Background}
The background of this study include
\section{Near Infrared Spectroscopy (NIRS)}
..............................\par
.................
\subsection{Introduction}
\section{Metabolomics}
\subsection{Introduction}
\subsection{Mass Spectrometry}


\section{Machine Learning}
\subsection{Introduction}
\subsection{Partial Least Square Regression (PLS)}
\subsection{Random Forest (RF)}
\subsection{Convolutional Neural Network (CNN)}



\chapter{Methods and Implementation}
\section{MSNovelist}
\subsection{Basic Idea of the Software}
\subsection{Deep Learning Architecture}
\subsection{Preparation of Data}

\section{Data}
\subsection{Data Records}
\subsection{File Formats}

\section{Evaluation of the Training}

\section{Implementations in the TensorFlow Framework}
\subsection{Preprocessing of Data}
\subsection{Training}
\subsection{Evaluation}
\subsection{Termination Criterion}
\subsection{Training on the Pubchem Dataset}

\section{Implementations in the PyTorch Framework}
\subsection{Reimplementation of the LSTM Architecture}
\subsection{Training with DeepSMILES and SELFIES}
\subsection{Transformer Implementation}

\section{Software}
\section{Hardware}

\chapter{Results and Discussion}
\section{Baseline Model}
\subsection{Termination Criterion}
\subsection{Impact of the Amount of Data}
\subsection{Duplicates in the Fingerprints}

\section{Alternative Tokenization of SMILES Sequences}
\subsection{QBMG Tokenization}

\end{document}
