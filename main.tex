\documentclass[12pt,a4paper]{report}
\usepackage{graphicx} % For including images
\usepackage{tocloft} % Optional: Customize the table of contents
\usepackage[a4paper, margin=2cm]{geometry} % Adjust margin 
\setcounter{secnumdepth}{3} % Allows subsubsections to be numbered
\setcounter{tocdepth}{3} % Includes up to subsubsections in the TOC

\begin{document}

% Title Page
\begin{titlepage}
    \centering
    \begin{flushright}
        \centering
        \includegraphics[width=2cm]{images/thd.png} % Adjust width as needed
        \hspace{0cm} % Space between images
        \includegraphics[width=2cm]{images/ipb.jpg} % Adjust width as needed
    \end{flushright}
    {\huge\bfseries Development of an R Toolbox for Near-Infrared Spectroscopy Data Processing \\
    and Analysis of Plant Metabolic Phenotypes\par}
    \vspace{2cm}
    {\LARGE \textsc{Deggendorf Institute of Technology}\par}
    \vspace{1cm}
    {\Large \textsc{MSc. Life Science Informatics}\par}
    \vspace{1.5cm}
    {\Large\itshape Methun George\par}
    \vfill
    Supervised by\par
    PD Dr. habil. rer. nat. Steffen Neumann\par
    Prof. Dr. Melanie Kappelmann-Fenzl
    \vfill
    {\large \today\par}
\end{titlepage}

% Table of Contents
\newpage
\tableofcontents
\newpage

% Chapters and Sections

\chapter{Introduction}
The advancement and widespread use of heigh-throughput experimental 
technologies in the field of plant biology have introduced significant 
challenges in managing and analysing the vast datasets effectively. Addressing
these challenges require innovative methods that maximize the data utility 
while mimnimizing computational inefficiencies and resource consumption, ensuring
robust insights into complex biological systems (https://www.frontiersin.org/research-topics/6856/machine-learning-in-plant-science/articles).



The understanding of interplay between plant physiology and its hidden biochemical process is crucial for the improvement of basic plant science and addressing global challenges such as food security, crop resilience and combating climate change [1]. 
In recent years, advanced High-throughput analytical techniques such as Near-Infrared Spectroscopy (NIRS) and Liquid Chromatography-Mass Spectrometry (LC-MS) has instigated a paradigm shift in plant biology [2][3].
These High-throughput techniques are mostly used in areas like genomics, imaging and spectroscopy and is known for their ability to collect and analyse the data faster than traditional techniques[3].

\chapter{Background}
The background of this study include
\section{Related Work}
\section{Near Infrared Spectroscopy (NIRS)}
\section{R Programming}
\section{Machine Learning}
\subsection{Partial Least Square Regression (PLSR)}
\subsection{Random Forest (RF)}
\subsection{Convolutional Neural Network (CNN)}
\section{Mass Spectrometry and Liquid Chromatography}



\chapter{Implementation}
\section{Packages}
Github, testing, actions
\section{Contributions elsewhere}
\section{HPC runs}


\chapter{Results and Discussion}

\section{Data charecterestics}
histogram, spectra
\section{Baseline Machine Learning Models Pablo}
PLS, RF, CNN

\subsection{Variable importance}

\section{ Variations in Baseline systems}
\subsection{ modifying the Test and Training split}
\subsection{ input data length}

\section{Sues}

\chapter{Reference}
1. Pieruschka R, Schurr U. Plant Phenotyping: Past, Present, and Future. Plant Phenomics. 2019 Mar 26;2019:7507131. doi: 10.34133/2019/7507131. PMID: 33313536; PMCID: PMC7718630.
2. Pulok K. Mukherjee, Quality control and evaluation of herbal drugs, Evaluating natural products and traditional medicine. 2019, doi:10.1016/C2016-0-042328, ISBN:978-0-12-813374-3
3. Nizamani, M. M., Zhang, Q., Muhae-Ud-Din, G., & Wang, Y. (2023). High-throughput sequencing in plant disease management: A comprehensive review of benefits, challenges, and future perspectives. Phytopathology Research, 5(44). https://doi.org/10.1186/s42483-023-00215-7
4. Lane, H. M., & Murray, S. C. (2021). High throughput can produce better decisions than high accuracy when phenotyping plant populations. Crop Science, 61(3), 1473–1484. https://doi.org/10.1002/csc2.20514



\end{document}
