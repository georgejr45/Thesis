\documentclass[12pt,a4paper]{report}
\usepackage{graphicx} % For including images
\usepackage{tocloft} % Optional: Customize the table of contents
\usepackage[a4paper, margin=2cm]{geometry} % Adjust margin 
\setcounter{secnumdepth}{3} % Allows subsubsections to be numbered
\setcounter{tocdepth}{3} % Includes up to subsubsections in the TOC

\begin{document}

% Title Page
\begin{titlepage}
    \centering
    \begin{flushright}
        \centering
        \includegraphics[width=2cm]{images/thd.png} % Adjust width as needed
        \hspace{0cm} % Space between images
        \includegraphics[width=2cm]{images/ipb.jpg} % Adjust width as needed
    \end{flushright}
    {\huge\bfseries Development of an R Toolbox for Near-Infrared Spectroscopy Data Processing \\
    and Analysis of Plant Metabolic Phenotypes\par}
    \vspace{2cm}
    {\LARGE \textsc{Deggendorf Institute of Technology}\par}
    \vspace{1cm}
    {\Large \textsc{MSc. Life Science Informatics}\par}
    \vspace{1.5cm}
    {\Large\itshape Methun George\par}
    \vfill
    Supervised by\par
    PD Dr. habil. rer. nat. Steffen Neumann\par
    Prof. Dr. Melanie Kappelmann-Fenzl
    \vfill
    {\large \today\par}
\end{titlepage}

% Table of Contents
\newpage
\tableofcontents
\newpage

% Chapters and Sections

\chapter{Introduction}

The understanding of interplay between plant physiology and its hidden biochemical process is crucial for the improvement of basic plant science and addressing global challenges such as food security, crop resilience and combating climate change [1]. 
In recent years, advanced High-throughput analytical techniques such as Near-Infrared Spectroscopy (NIRS) and Liquid Chromatography-Mass Spectrometry (LC-MS) has instigated a paradigm shift in plant biology [2][3].
These High-throughput techniques are mostly used in areas like genomics, imaging and spectroscopy and is known for their ability to collect and analyse the data faster than traditional techniques[3].
High-throughput techniques are widely used since they enable the efficient collection of vast amount of data at various scales, from molecular to field level over significant time periods[4].
The big data generated by these high throughput procedures present both opportunities and challenges at the same time. It requires efficient processing to extract maximum useful results and this is where Machine Learning (ML) or Deep Learning (DL) becomes indispensable [4][5].
ML as a part of Artificial Intelligence (AI) refers to the ability of computers to find patterns and learn from the existing data which can be employed in processing high dimensional data [6][4]. The ML algorithms are powerful enough to analyse complex, high dimensional datasets, 
enabling accurate predictions of plant traits or other features based on the input data. Additionally, integrating these big data with ML could help the researchers to optimize data processing pipelines, enhance predictive accuracy and thereby enter into a new era of data-driven decision-making [4][5]. This project employs linear models, non-linear models and neural networks to predict various plant features and compare their performances. \\


A significant  shift in the realm of the biomedical community has brought new guidelines to ensure readability, modularity, transparency and extensibility of computational toolboxes. A toolbox, which stores multiple functions, parameters and results in a central location should be maintainable and uncomplicated for the developers and members of the open-source community [7].
R is a powerful and widely used programming language in the analysis and processing of high throughput data. Additionally, R contains a multitude of statistical and high quality visualization packages such as ggplot2 which are capable of processing and integrating big data to different ML methods [8]. Bioconductor is an open source R software for bioinformatics, which contains more than 3000 packages for statistical computing. 
This offers an object oriented framework for the high dimensional data, cutting edge visualization capabilities and interoperability [9]. Existing tools in Near-Infrared Spectroscopy (NIRS) data processing lack functionalities that could simplify and standardize data workflows when integrated with the SummarizedExperiment framework from the Bioconductor package. To address these gaps, the R toolbox, “nearspectRa” was developed for processing NIRS data.
This package has a modular structure which creates a SummarizedExperiment object from NIRS data. \\


Metabolomics, the study of small molecular compounds in biological systems, is a rapidly advancing field of science with applications in biotechnology, medicine, synthetic biology and environmental science [6]. Metabolomics has emerged as a transformative tool in plant biology, enabling cost-efficient and high throughput molecular characterization.
The integration of metabolomics with different omics approaches has proven invaluable for functional genes identification and developing trait specific markers [10]. Metabolomics, which is built on the advancement of phenomics and genomics, provides high throughput and precise profiling of metabolites, revealing the physiological state of cells [6][10].
Metabolites play a crucial role in plant metabolism, influencing its biomass and architecture therefore study of these small molecules will aid in uncovering plant regulatory mechanisms and pathway interactions [10].
The coupling of liquid or gas chromatography with mass spectrometry or nuclear magnetic resonance spectroscopy (NMR) facilitates measurement of thousands of metabolites, thereby providing a comprehensive view of biochemical and biological mechanisms [11]. Therefore, Mass spectrometry(MS) remains the most widely used analytical approach among others due to its versatility and sensitivity [6].
Mass spectrometry based metabolomics generate data of high sensitivity and throughput requiring advanced computational methods. Machine learning not only offers a powerful solution to analyse such data, but also helps in resolving the challenges like noise, batch effects and missing values [12]. Integrating ML with Liquid Chromatography-Mass Spectroscopy (LC-MS) data helps us to analyse this complex heterogeneous data rapidly, enabling deeper insights. \\


Near-Infrared Spectroscopy (NIRS) is an advanced high throughput and non-destructive analytical technique that uses light in the near-infrared region (700-2500 nm) to assess the chemical composition of samples [13]. The light is either absorbed or reflected by the sample at different wavelengths and thereby creating a spectrum[13]. The NIRS is widely used in plant research due to its ability in predicting sample structure and traits by analysing the spectral patterns.
NIRS can also be used in the quantitative analysis of key plant features such as protein and carbohydrate content, secondary metabolites and physiological traits such as Specific Leaf Area (SLA) by developing calibration models between spectra and reflectance trait data [14][13]. The NIRS is not only used in plant biology but also in various fields such as food science, agriculture and pharmaceuticals. When compared to other analytical techniques, NIRS is rapid, 
requires minimal sample preparation and less expensive, which makes it more attractive and interesting to the scientific communities [13]. However, on the flip side it requires complex statistical methods to extract different complex features due to the highly-correlated nature of NIRS data [13]. To tackle this problem, the conventional methods such as Partial Least Square Regression (PLSR) and Principal Component Analysis (PCA) imply dimension reduction which 
result in loss of information and often struggles to extract important features from the spectral data [13]. To address the challenge of data complexity and generalizability, different ML methods can be used to predict the traits from the NIRS data [13][15]. In this project different ML and Deep Learning (DL) has been employed to predict different plant leaf traits with use of NIRS data. \\


In recent years, studies using NIRS data coupled with PLSR have been used as an alternative for traditional methods such as high-performance liquid chromatography (HPLC) and mass spectrometry which are both labor-intensive and expensive to predict different plant traits. A notable example is the prediction of glucobrassicin (GBS) concentrations from NIRS data. This has shown that GBS 
concentrations could be reliably predicted from NIRS data [16]. Another prominent example is the tree and mycorrhizal fungal diversity experiment and  trait variation in temperate forests  conducted by Pablo Castro Sanchez-Bermejo, where he combined Deep Learning (DL) approaches with leaf-level spectral data to predict 5 different leaf traits [15]. Another good  example would be a 
project involving the development of white-box  workflow for regression tasks [17]. The project marks the potential of Regression (Sensitive) Neural Gas (RSNG) for generating interpretable results while maintaining high accuracy [17]. From the above studies, it is evident that NIRS data has a wide range of applications in plant research. This can also be expanded further to predict 
complex metabolites which are usually assessed via techniques like LC-MS. Moreover, integrating NIRS with advanced ML could further enhance the prediction accuracy and unlock new possibilities in plant science. \\


The past decade has witnessed the increasing popularity of  Artificial Intelligence (AI) in different fields. However, this idea of AI has been under development since 1956, starting from the concept of “programming computers to think and reason” [18]. In other words, AI can be described as “automating intellectual tasks normally done by humans” [18]. Machine learning (ML) and Deep learning (DL) are the methods that fall under the realm of AI [18][19]. 
Nowadays, there are different ML algorithms in use, in which the most popular ones include Partial Least Square Regression (PLSR), Random Forest (RF) and Convolutional Neural Network (CNN) [20][18]. PLSR is a linear and most simple ML approach. It uses a straight line to solve the regression problem in the high dimensional data [18]. On the other hand,  Random forest is a non-linear approach in ML that is primarily used for classification. It can also 
be used for regression tasks and can be represented as a decision tree with a series of nodes starting from a root node. The terminal node will predict the class of data [18]. The Convolutional neural networks are a specialized type of neural network which are mainly used in the field of image processing [21]. A recent study of mycorrhizal fungal diversity experiment and  trait variation in temperate forests  conducted by Pablo Castro Sanchez-Bermejo, 
demonstrated the application of CNN in predicting the leaf trait values from NIRS data, achieving superior results [15]. This outcome strongly suggests the potential of CNN not only for classification tasks such as image processing but also for regression tasks. In another project, CNN was used to predict gene expression status on the basis of sequence of gene transcription start regions. The CNN model had achieved roughly 80% accuracy [22]. 
These studies highlight the growing versatility of CNN models. \\


Omics is a term associated with the field of large scale biological data, including genomics, epigenomics, proteomics, transcriptomics and metabolomics [23]. Combination of data from these techniques along with advanced microscopy techniques helps in the study of biomolecules in cellular and subcellular levels [23]. However, the high throughput data from these omics instruments poses challenges in processing and analysing it without the loss of information [23][10]. 
The complexity and scale of this data make ML essential for effective integration and analysis, raising the critical question: which ML model is best suited to handle this data? How much programming expertise is necessary to implement these models? And, which models are most suitable for regression tasks?. Each ML model handles the data differently. For instance, PLSR uses latent variables to capture the covariance between predictor and response variables. Moreover, 
PLSR uses the combination of Principal Component Analysis (PCA) and linear regression [20]. In the case of RF, it follows the concept of “a forest made of many trees” which uses the combination of predictions from many trees [18]. Among these ML techniques, CNN is gaining attention on its ability in handling high throughput data and predicting with remarkable accuracy [15][22]. These ML models also require different levels of programming proficiency and computational resources, depending on the scale of data. \\


In the light of the findings, it is clear that ML can significantly improve the analysis and processing of high throughput data from analytical techniques such as LC-MS and NIRS [15] [22] [17]. Among these, NIRS stands out as a non-destructive, cost-effective and rapid method, offering valuable insights into the chemical composition of the biological samples [12] [13] [14]. These qualities make NIRS a promising technique to optimize and integrate with ML and DL models for predictive accuracy. \\


Given the popularity of R programming within the ecological and bioinformatic community, it was chosen as the foundation for this project [7] [9]. Recognizing the need for specialized tools to process the NIRS data, an R package, nearspectRa, was developed to handle data from two widely used NIRs instruments namely “ASD Fieldspec 4” and Spectra Vista Corporation (SVC) HR-1024i. 
Leveraging supporting packages like “R-FieldSpectra”, the high dimensional data was structured into a “SummarizedExperiment” object, aligning with Bioconductor standards for interoperability and integration. \\


Apart from developing a Good Scientific Practice (GSP) compliant package, this project involved two key analyses: first, predicting plant leaf traits from NIRS data using three popular ML methods, PLSR, RF and CNN and second predicting LC-MS features from NIRS data using the same models. To evaluate these approaches, performance was compared using metrics such as the coefficient of determination (R²), Root Mean Squared Error (RMSE) and training time of each model. Additionally,
extrapolation studies were conducted on PLSR and RF to assess the robustness and performance of those beyond training data. This project not only exemplifies good scientific practice in developing an R toolbox but also provides a comprehensive comparison of linear, non-linear and neural network based NL approaches in predicting plant traits and LC-MS features from NIRS data. By achieving this, the project makes a significant milestone, paving the way for a new era of cost-effective, rapid biochemical analysis in metabolomics. \\



\chapter{Background}
The background of this study include
\section{Related Work}
\section{Near Infrared Spectroscopy (NIRS)}
\section{R Programming}
\section{Machine Learning}
\subsection{Partial Least Square Regression (PLSR)}
\subsection{Random Forest (RF)}
\subsection{Convolutional Neural Network (CNN)}
\section{Mass Spectrometry and Liquid Chromatography}



\chapter{Implementation}
\section{Packages}
Github, testing, actions
\section{Contributions elsewhere}
\section{HPC runs}


\chapter{Results and Discussion}

\section{Data charecterestics}
histogram, spectra
\section{Baseline Machine Learning Models Pablo}
PLS, RF, CNN

\subsection{Variable importance}

\section{ Variations in Baseline systems}
\subsection{ modifying the Test and Training split}
\subsection{ input data length}

\section{Sues}

\chapter{Reference}
1. Pieruschka R, Schurr U. Plant Phenotyping: Past, Present, and Future. Plant Phenomics. 2019 Mar 26;2019:7507131. doi: 10.34133/2019/7507131. PMID: 33313536; PMCID: PMC7718630. \\
2. Pulok K. Mukherjee, Quality control and evaluation of herbal drugs, Evaluating natural products and traditional medicine. 2019, doi:10.1016/C2016-0-042328, ISBN:978-0-12-813374-3. \\
3. Nizamani, M. M., Zhang, Q., Muhae-Ud-Din, G., & Wang, Y. (2023). High-throughput sequencing in plant disease management: A comprehensive review of benefits, challenges, and future perspectives. Phytopathology Research, 5(44). https://doi.org/10.1186/s42483-023-00215-7. \\
4. Lane, H. M., & Murray, S. C. (2021). High throughput can produce better decisions than high accuracy when phenotyping plant populations. Crop Science, 61(3), 1473–1484. https://doi.org/10.1002/csc2.20514. \\
5. Zhang, N., Zhou, X., Kang, M., Hu, B.-G., Heuvelink, E., & Marcelis, L. F. M. (2023). Machine learning versus crop growth models: An ally, not a rival. Journal of Experimental Botany, 74(4), 1259–1276. https://doi.org/10.1093/jxb/erac517. \\
6. Zhu H. Big Data and Artificial Intelligence Modeling for Drug Discovery. Annu Rev Pharmacol Toxicol. 2020 Jan 6;60:573-589. doi: 10.1146/annurev-pharmtox-010919-023324. Epub 2019 Sep 13. PMID: 31518513; PMCID: PMC7010403. \\
7. Kelsey Chetnik, Elisa Benedetti, Daniel P Gomari, Annalise Schweickart, Richa Batra, Mustafa Buyukozkan, Zeyu Wang, Matthias Arnold, Jonas Zierer, Karsten Suhre, Jan Krumsiek,  maplet: an extensible R toolbox for modular and reproducible metabolomics pipelines, Bioinformatics, Volume 38, Issue 4, February 2022, Pages 1168–1170, https://doi.org/10.1093/bioinformatics/btab741. \\
8. Peng, Roger D. R programming for data science. Victoria, BC, Canada: Leanpub, 2016. \\
9. www.bioconductor.org \\
10. Kumar, R., Bohra, A., Pandey, A. K., Pandey, M. K., & Kumar, A. (2017). Metabolomics for plant improvement: Status and prospects. Frontiers in Plant Science, 8, 1302. https://doi.org/10.3389/fpls.2017.01302. \\
11. González-Domínguez, R., García-Barrera, T., & Gómez-Ariza, J. L. (2014). Metabolite profiling for the identification of altered metabolic pathways in Alzheimer's disease. Journal of Pharmaceutical and Biomedical Analysis, 107, 75–81. https://doi.org/10.1016/j.jpba.2014.10.010. \\
12. Liebal, U. W., Phan, A. N., Sudhakar, M., Raman, K., & Blank, L. M. (2020). Machine learning applications for mass spectrometry-based metabolomics. Metabolites, 10(6), 243. https://doi.org/10.3390/metabo10060243. \\
13. Vaillant, A., Beurier, G., Cornet, D., Rouan, L., Vile, D., Violle, C., & Vasseur, F. (2024). NIRSpredict: a platform for predicting plant traits from near infra-red spectroscopy. BMC Plant Biology, 24(1), 1100. https://link.springer.com/article/10.1186/s12870-024-05776-0. \\
14. Marr S, Hageman JA, Wehrens R, van Dam NM, Bruelheide H, Neumann S. LC-MS based plant metabolic profiles of thirteen grassland species grown in diverse neighbourhoods. Sci Data. 2021 Feb 9;8(1):52. doi: 10.1038/s41597-021-00836-8. PMID: 33563993; PMCID: PMC7873126. \\
15. Sánchez-Bermejo, P. C., Monjau, T., Goldmann, K., Ferlian, O., Eisenhauer, N., Bruelheide, H., Ma, Z., & Haider, S. (2024). Tree and mycorrhizal fungal diversity drive intraspecific and intraindividual trait variation in temperate forests: Evidence from a tree diversity experiment. Functional Ecology. https://doi.org/10.1111/1365-2435.14549. \\
16. Renner, I.E., Fritz, V.A. Using Near-infrared reflectance spectroscopy (NIRS) to predict glucobrassicin concentrations in cabbage and brussels sprout leaf tissue. Plant Methods16, 136 (2020). https://doi.org/10.1186/s13007-020-00681-7. \\
17. R. Schubert et al., "A White-Box Workflow for the Prediction of Food Content From Near-Infrared Data Based on Fourier-Transformation," 2023 13th Workshop on Hyperspectral Imaging and Signal Processing: Evolution in Remote Sensing (WHISPERS), Athens, Greece, 2023, pp. 1-5, doi: 10.1109/WHISPERS61460.2023.10430694. \\
18. Choi RY, Coyner AS, Kalpathy-Cramer J, Chiang MF, Campbell JP. Introduction to Machine Learning, Neural Networks, and Deep Learning. Transl Vis Sci Technol. 2020 Feb 27;9(2):14. doi: 10.1167/tvst.9.2.14. PMID: 32704420; PMCID: PMC7347027.\\
19. Pichler, M., & Hartig, F. (2022). Machine learning and deep learning—A review for ecologists. Methods in Ecology and Evolution, 13(9), 1984–2000. https://doi.org/10.1111/2041-210X.14061. \\
20. Ji, F., Li, F., Hao, D., Shiklomanov, A. N., Yang, X., Townsend, P. A., Dashti, H., Nakaji, T., Kovach, K. R., Liu, H., Luo, M., & Chen, M. (2024). Unveiling the transferability of PLSR models for leaf trait estimation: Lessons from a comprehensive analysis with a novel global dataset. New Phytologist. https://doi.org/10.1111/nph.19807. \\
21. Georgios Kourounis, Ali Ahmed Elmahmudi, Brian Thomson, James Hunter, Hassan Ugail, Colin Wilson, Computer image analysis with artificial intelligence: a practical introduction to convolutional neural networks for medical professionals, Postgraduate Medical Journal, Volume 99, Issue 1178, December 2023, Pages 1287–1294, https://doi.org/10.1093/postmj/qgad095. \\
22. Liu S, Cheng H, Ashraf J, Zhang Y, Wang Q, Lv L, He M, Song G, Zuo D. Interpretation of convolutional neural networks reveals crucial sequence features involving in transcription during fiber development. BMC Bioinformatics. 2022 Mar 15;23(1):91. doi: 10.1186/s12859-022-04619-9. PMID: 35291940; PMCID: PMC8922751. \\
23. van Dijk ADJ, Kootstra G, Kruijer W, de Ridder D. Machine learning in plant science and plant breeding. iScience. 2020 Dec 5;24(1):101890. doi: 10.1016/j.isci.2020.101890. PMID: 33364579; PMCID: PMC7750553. \\

\end{document}
